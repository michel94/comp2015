
\documentclass[12pt]{article}
\usepackage[english]{babel}
\usepackage[utf8x]{inputenc}
\usepackage{amsmath}
\usepackage{graphicx}
\usepackage[colorinlistoftodos]{todonotes}

\begin{document}

\begin{titlepage}

\newcommand{\HRule}{\rule{\linewidth}{0.5mm}} % Defines a new command for the horizontal lines, change thickness here

\center
 
%----------------------------------------------------------------------------------------
%	HEADING SECTIONS
%----------------------------------------------------------------------------------------

\textsc{\LARGE University of Coimbra}\\[1.5cm] % Name of your university/college
\textsc{\Large Bachelor in Informatics Engineering}\\[0.5cm] % Major heading such as course name
\textsc{\large Compilers Course}\\[0.5cm] % Minor heading such as course title

%----------------------------------------------------------------------------------------
%	TITLE SECTION
%----------------------------------------------------------------------------------------

\HRule \\[0.4cm]
{ \huge \bfseries Compilers Project Report}\\[0.4cm] % Title of your document
\HRule \\[1.5cm]
 
%----------------------------------------------------------------------------------------
%	AUTHOR SECTION
%----------------------------------------------------------------------------------------

\begin{minipage}{0.4\textwidth}
\begin{flushleft} \large
\emph{Authors:}\\
Ricardo Gomes \\
Miguel Duarte
\end{flushleft}
\end{minipage}
~
\begin{minipage}{0.4\textwidth}
\begin{flushright} \large
\end{flushright}
\end{minipage}\\[2cm]

%----------------------------------------------------------------------------------------
%	DATE SECTION
%----------------------------------------------------------------------------------------

{\large \today}\\[2cm] % Date, change the \today to a set date if you want to be precise

%----------------------------------------------------------------------------------------
%	LOGO SECTION
%----------------------------------------------------------------------------------------

\includegraphics [width=9cm]{logo.jpg}\\[1cm] % Include a department/university logo - this will require the graphicx package
 
%----------------------------------------------------------------------------------------

\vfill % Fill the rest of the page with whitespace

\end{titlepage}


\begin{abstract}
Your abstract. REKT
\end{abstract}

\section{Introduction}

Your introduction goes here! Some examples of commonly used commands and features are listed below, to help you get started.

If you have a question, please use the support box in the bottom right of the screen to get in touch. 

\newpage

\section{Lexical Analysis}

Lexical analysis was really simple and few lines of code were written for this phase. Next we refer the more important things that we did.\\

\subsection{Options}

We enabled \'caseless\' and \'yylineno\'. Caseless is used to match both lower case and upper case in the regular expressions, because pascal is a case insensitive language. Yylineno is used to tell lex to update the yylineno variable with current line of the token read.

\subsection{YY\_USER\_ACTION}

We defined a YY\_USER\_ACTION, that runs code for each token read. It's used to increment the current column variable and to update yylloc and yyval lex/yacc shared variables. The default yylloc struct was used, with the properties first\_line and first\_column. We found this a easy way to access the column and the line of each token, which is useful in later phases, for example in semantic analysis. We also store the string of the token in the yyval.str.\\

\subsection{Comment Start States}

The case of comments parsing is a bit more complicated. Because of that we used a start state that works similar to a state a machine.\\
Basically we define different states and when something is matched we jump to other states or we do some updates like updating column number.\\
For Pascal comments when `{' or `(*' is parsed we define a begin state and until we read `}' or `*)' we do nothing but control column number.\\

\subsection{Basic Tokens}

The tokens are parsed in a straight forward way. If the regular expression matches a token, returns the correspondent token id, generated by yacc.\\ For the operators made only by one character, we use the same regular expression and return the first character read.\\

\subsection{Errors}

Unterminated string error is caught by reading a specific regular expression right after the terminated string r.e. So, if the string is not read as terminated, it will be read as unterminated, at end of the line, because the \textbackslash n character is not a valid one.\\
Unterminated comment error is caugnt when eof is read inside the comment start state.\\
Error handling is parsed by printing the invalid character and the respective column and line (using our column variable and yylineno).\\

\newpage

\section{Some \LaTeX{} Examples}
\label{sec:examples}

\subsection{Sections}

Use section and subsection commands to organize your document. \LaTeX{} handles all the formatting and numbering automatically. Use ref and label commands for cross-references.

\subsection{Comments}

Comments can be added to the margins of the document using the \todo{Here's a comment in the margin!} todo command, as shown in the example on the right. You can also add inline comments too:

\todo[inline, color=green!40]{This is an inline comment.}

\subsection{Tables and Figures}

Use the table and tabular commands for basic tables --- see Table~\ref{tab:widgets}, for example. You can upload a figure (JPEG, PNG or PDF) using the files menu. To include it in your document, use the includegraphics command as in the code for Figure~\ref{fig:frog} below.

% Commands to include a figure:
\begin{figure}
\centering
\includegraphics[width=0.5\textwidth]{frog.jpg}
\caption{\label{fig:frog}This is a figure caption.}
\end{figure}

\begin{table}
\centering
\begin{tabular}{l|r}
Item & Quantity \\\hline
Widgets & 42 \\
Gadgets & 13
\end{tabular}
\caption{\label{tab:widgets}An example table.}
\end{table}

\subsection{Mathematics}

\LaTeX{} is great at typesetting mathematics. Let $X_1, X_2, \ldots, X_n$ be a sequence of independent and identically distributed random variables with $\text{E}[X_i] = \mu$ and $\text{Var}[X_i] = \sigma^2 < \infty$, and let
$$S_n = \frac{X_1 + X_2 + \cdots + X_n}{n}
      = \frac{1}{n}\sum_{i}^{n} X_i$$
denote their mean. Then as $n$ approaches infinity, the random variables $\sqrt{n}(S_n - \mu)$ converge in distribution to a normal $\mathcal{N}(0, \sigma^2)$.

\subsection{Lists}

You can make lists with automatic numbering \dots

\begin{enumerate}
\item Like this,
\item and like this.
\end{enumerate}
\dots or bullet points \dots
\begin{itemize}
\item Like this,
\item and like this.
\end{itemize}

We hope you find write\LaTeX\ useful, and please let us know if you have any feedback using the help menu above.

\end{document}