
\documentclass[12pt]{article}
\usepackage[english]{babel}
\usepackage[utf8x]{inputenc}
\usepackage{amsmath}
\usepackage{graphicx}
\usepackage[colorinlistoftodos]{todonotes}

\begin{document}

\begin{titlepage}

\newcommand{\HRule}{\rule{\linewidth}{0.5mm}} % Defines a new command for the horizontal lines, change thickness here

\center
 
%----------------------------------------------------------------------------------------
%	HEADING SECTIONS
%----------------------------------------------------------------------------------------

\textsc{\LARGE University of Coimbra}\\[1.5cm] % Name of your university/college
\textsc{\Large Bachelor in Informatics Engineering}\\[0.5cm] % Major heading such as course name
\textsc{\large Compilers Course}\\[0.5cm] % Minor heading such as course title

%----------------------------------------------------------------------------------------
%	TITLE SECTION
%----------------------------------------------------------------------------------------

\HRule \\[0.4cm]
{ \huge \bfseries Compilers Project Report}\\[0.4cm] % Title of your document
\HRule \\[1.5cm]
 
%----------------------------------------------------------------------------------------
%	AUTHOR SECTION
%----------------------------------------------------------------------------------------

\begin{minipage}{0.4\textwidth}
\begin{flushleft} \large
\emph{Authors:}\\
Ricardo Gomes \\
Miguel Duarte
\end{flushleft}
\end{minipage}
~
\begin{minipage}{0.4\textwidth}
\begin{flushright} \large
\end{flushright}
\end{minipage}\\[2cm]

%----------------------------------------------------------------------------------------
%	DATE SECTION
%----------------------------------------------------------------------------------------

{\large \today}\\[2cm] % Date, change the \today to a set date if you want to be precise

%----------------------------------------------------------------------------------------
%	LOGO SECTION
%----------------------------------------------------------------------------------------

\includegraphics [width=9cm]{logo.jpg}\\[1cm] % Include a department/university logo - this will require the graphicx package
 
%----------------------------------------------------------------------------------------

\vfill % Fill the rest of the page with whitespace

\end{titlepage}


\section{Introduction}

This report concerns the implementation of a mili-pascal compiler and is a project for the Compilers course.

\subsection{Code organization}

Multiple code files were used for this project. mpacompiler.l has the lex code, for lexical analysis. mpacompiler.y has the yacc code and some code for parsing tree generation, used when yyparse runs. parsing.h has basically the code for semantic analysis. Includes parsing of the generated tree and creation of symbol tables.\\
The other two files have some support code. types.h define all data types, including the node struct, and provide functions for convertion between types, for example string to type, type to string, type to llvm type, etc. hashtable.f implement a hashtable, used to implement the symbol tables used.\\

\newpage

\section{Lexical Analysis}

Lexical analysis was really simple and few lines of code were written for this phase. Next we refer the more important things that we did.\\

\subsection{Options}

We enabled ``caseless" and ``yylineno". Caseless is used to match both lower case and upper case in the regular expressions, because pascal is a case insensitive language. Yylineno is used to tell lex to update the yylineno variable with current line of the token read.

\subsection{YY\_USER\_ACTION}

We defined a YY\_USER\_ACTION, that runs code for each token read. It's used to increment the current column variable and to update yylloc and yyval lex/yacc shared variables. The default yylloc struct was used, with the properties first\_line and first\_column. We found this a easy way to access the column and the line of each token, which is useful in later phases, for example in semantic analysis. We also store the string of the token in the yyval.str.\\

\subsection{Basic Tokens}

The tokens are parsed in a straight forward way. If the regular expression matches a token, returns the correspondent token id, generated by yacc.\\ For the operators made only by one character, we use the same regular expression and return the first character read.\\

\subsection{Comment Start States}

The case of comments parsing is a bit more complicated. Because of that we used a start state that works similarly to a state a machine.\\
Basically we define different states and when something is matched we jump to other states or we do some updates like setting column number.\\
For Pascal comments when ``\{" or ``(*" is parsed we define a begin state and until we read ``\}" or ``*)" we do nothing but control column number.\\

\subsection{Errors}

Unterminated string error is caught by reading a specific regular expression right after the terminated string r.e. So, if the string is not read as terminated, it will be read as unterminated, at end of the line, because the \textbackslash n character is not a valid one.\\
Unterminated comment error is caugnt when eof is read inside the comment start state.\\
Error handling is parsed by printing the invalid character and the respective column and line (using our column variable and yylineno).\\

\newpage

\section{Syntax Analysis}

\subsection{Token Types}

We define two types of yacc token: str (char*) and node (struct node*). \\
The tokens id, string, reallit and intlit where declared with type str, the reserved names without type and the operation symbols with type str (to be used later in semantic errors).\\

\subsection{Parsing Tree Nodes}

The nodes of the parsing tree contain multiple variables. ``value" and ``value2" contain the token string for terminal nodes, in lower case and as read in the file, respectively.\\
The variable ``op" and ``n\_op" are used to store child nodes and number of child nodes, respectively. ``to\_use" is used to identify a node as a list-node i.e., as a superfluous one, to be removed in the bottom-up parsing.\\

\subsection{Parsing}

Parsing is made using two functions: ``make\_node" and ``terminal". \\
``terminal" generates the node for the terminal symbols (id, intlit, realit and string). Obviously, those nodes are non-superfluous. Also, the value of the node is stored in lower case to be used later in id lookups.\\
The ``make\_node" function creates each other type of node, receiving the node type and the ``to\_use" variable in its arguments. The node arguments are parsed and add to this node children. If any of those children is a list-node, those child's children (grandchildren of the new node) are appended to the new node children.\\

\subsection{Shift/Reduce Conflicts}

After finising the translation of the given grammar in the project report from ebnf to bnf and write it to yacc we got some shift/reduce conflicts as expected, since the grammar is ambiguous. The problems were in the if/else and in the majority of the expressions. \\
The expressions were fixed by replacing their productions with the ones from the grammar provided by ISO-7185:1990. The famous dangling else conflict was resolved by giving right precedence to both `then' and `else', but giving greater precedence to the else token. \\
So, reducing a ``if then else" expression has greater precedence than reducing a ``if then" one. \\

\section{Some \LaTeX{} Examples}
\label{sec:examples}

\subsection{Sections}

Use section and subsection commands to organize your document. \LaTeX{} handles all the formatting and numbering automatically. Use ref and label commands for cross-references.

\subsection{Comments}

Comments can be added to the margins of the document using the \todo{Here's a comment in the margin!} todo command, as shown in the example on the right. You can also add inline comments too:

\todo[inline, color=green!40]{This is an inline comment.}

\subsection{Tables and Figures}

% Commands to include a figure:
\begin{table}
\centering
\begin{tabular}{l|r}
Item & Quantity \\\hline
Widgets & 42 \\
Gadgets & 13
\end{tabular}
\caption{\label{tab:widgets}An example table.}
\end{table}

\subsection{Mathematics}

\LaTeX{} is great at typesetting mathematics. Let $X_1, X_2, \ldots, X_n$ be a sequence of independent and identically distributed random variables with $\text{E}[X_i] = \mu$ and $\text{Var}[X_i] = \sigma^2 < \infty$, and let
$$S_n = \frac{X_1 + X_2 + \cdots + X_n}{n}
      = \frac{1}{n}\sum_{i}^{n} X_i$$
denote their mean. Then as $n$ approaches infinity, the random variables $\sqrt{n}(S_n - \mu)$ converge in distribution to a normal $\mathcal{N}(0, \sigma^2)$.

\subsection{Lists}

You can make lists with automatic numbering \dots

\begin{enumerate}
\item Like this,
\item and like this.
\end{enumerate}
\dots or bullet points \dots
\begin{itemize}
\item Like this,
\item and like this.
\end{itemize}

\end{document}